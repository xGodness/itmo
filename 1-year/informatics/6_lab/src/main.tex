\documentclass{article}
\usepackage[12pt]{extsizes}
\usepackage[T2A]{fontenc}
\usepackage[utf8]{inputenc}
\usepackage[english, russian]{babel}

\usepackage{amssymb}
\usepackage{amsfonts}
\usepackage{amsmath}
\usepackage{enumitem}
\usepackage{graphics}

\usepackage{lipsum}



\usepackage{geometry} % Меняем поля страницы
\geometry{left=1cm}% левое поле
\geometry{right=1cm}% правое поле
\geometry{top=2cm}% верхнее поле
\geometry{bottom=1cm}% нижнее поле


\usepackage{fancyhdr} % Headers and footers
\pagestyle{fancy} % All pages have headers and footers
\fancyhead{} % Blank out the default header
\fancyfoot{} % Blank out the default footer
\fancyhead[L]{ЦРОД $\bullet$ Математика}
\fancyhead[C]{\text{Геометрия. Теория.}}
\fancyhead[R]{10.11 --- 21.11.20} % Custom header text


%-----------------------------------------------------------------------------------------------

%\begin{document}\normalsize
\begin{document}\large


\begin{center}\Large
 \textbf{Окружность и вписанность}
\end{center}


\begin{enumerate}[label*=\protect\fbox{\arabic{enumi}}]

 \item []  {\bf Основные определения:}\large
 %\normalsize
 
 \item [] \textit{(2) Радиус --- отрезок, соединяющий центр окружности и точку на окружности.}
 \item [] \textit{(3) Хорда --- отрезок, соединяющий две точки на окружности.}
 \item [] \textit{(4) Диаметр --- хорда, проходящая через центр окружности.}
 \item [] \textit{(5) Касательная --- прямая, имеющая с окружностью одну общую точку.}
 \item [] \textit{(6) Секущая --- прямая, имеющая с окружностью две общие точки.}
 \item [] \textit{(7) Дуга --- часть окружности, ограниченная двумя точками. Градусная мера дуги равна градусной мере соответственного центрального угла.}
 \item [] \textit{(8) Центральный угол --- угол, вершина которого лежит в центре окружности.}
 \item [] \textit{(9) Вписанный угол --- угол, вершина которого лежит на окружности, а стороны пересекают окружность.}
 \item [] \textit{(10) Биссектриса угла --- луч, проведенный из вершины угла и делящий его пополам.}
 \item [] \textit{(11) Серединный перпендикуляр --- прямая, проходящая через середину отрезка и перпендикулярная ему.}
 \item [] \textit{(12) Вписанная окружность --- окружность, для которой стороны многоугольника являются касательными.}
 \item [] \textit{(13) Описанная окружность --- окружность, проходящая через все вершины многоугольника.}
 \item []  {\bf Основные теоремы:}\large
 %\normalsize
 \item [] \textbf{\textit{Свойство касательной:}} \textit{касательная перпендикулярна радиусу, проведенному в точку касания.}
 \item [] \textbf{\textit{Свойство отрезков касательных:}} \textit{отрезки касательных, проведенных из одной точки, равны и образуют равные углы с прямой, проходящей через данную точку и центр окружности.}
 \item [] \textbf{\textit{Градусаная мера вписанного угла:}} \textit{градусная мера вписанного угла равна половине градусной меры дуги, на которую он опирается.}
  \item [] \textbf{\textit{Градусная мера угла между касательной и хордой:}} \textit{градусная мера угла между касательной и хордой равна половине градусной меры дуги, заключенной между ними.}
  \item [] \textbf{\textit{Равные дуги и равные хорды:}} \textit{равные хорды стягивают равные дуги.}
 \item [] \textbf{\textit{Свойство пересекающихся хорд:}} \textit{если две хорды одной окружности пересекаются, то произведение отрезков одной хорды равно произведению отрезков другой.}
 \item [] \textbf{\textit{Свойство отрезков секущих:}} \textit{если из точки, не лежащей на данной окружности, провести две секущие, то произведение отрезка одной секущей на ее внешнюю часть равно произведению отрезка другой секущей на ее внешнюю часть.}
  \item [] \textbf{\textit{Свойство отрезка касательной и секущей:}} \textit{если из точки, не лежащей на данной окружности, провести касательную и секущую, то квадрат отрезка касательной равен произведению секущей на ее внешнюю часть.}
  \item [] \textbf{\textit{Свойство точки, лежащей на биссектрисе угла:}} \textit{точка, лежащая на биссектрисе угла, равноудалена от сторон угла.}
  \item [] \textbf{\textit{Свойство точки, лежащей на серединном перпендикуляре к отрезку:}} \textit{точка, лежащая на серединном перпендикуляре к отрезку, равноудалена от его концов.}
  \item [] \textbf{\textit{Свойство центра вписанной окружности:}} \textit{центр вписанной в многоугольник окружности находится в точке пересечения биссектрис многоугольника.}
  \item [] \textbf{\textit{Свойство центра описанной окружности:}} \textit{центр описанной около многоугольника окружности находится в точке пересечения серединных перпендикуляров к сторонам многоугольника.}
  \item [] \textbf{\textit{Вписанный четырехугольник:}} \textit{четырехугольник вписан в окружность тогда и только тогда, когда сумма противоположных углов четырехугольника равна $180^{\circ}$.}
  \item [] \textbf{\textit{Описанный четырехугольник:}} \textit{в четырехугольник можно вписать окружность тогда и только тогда, когда суммы противоположных сторон четырехугольника равны.}
  \item []  {\bf Пара полезных следствий:}\large
  %\normalsize
  \item [] \textit{Вписанные углы, опирающиеся на одну дугу, равны.}
  \item [] \textit{Вписанный угол, опирающийся на диаметр, равен $90^{\circ}$.}
  
  
 

 
 
 
 %\item Сколько будет $2 + 2$?
 
 %\item $\cos(x)=\sin(x)$
 
 %\item $\log_23$



%\item 
%  \begin{enumerate}
%  \item[1] В группе из четырёх человек, говорящих на разных языках, любые трое могут общаться (возможно, один переводит двум другим). Доказать, что их можно разбить на пары, в каждой из которых имеется общий язык. 
%  \item[(2)] То же для группы из 100 человек. 
%  \item[3.] То же для группы из 102 человек.
% \end{enumerate}
 
 
 

% \item $a+b=c,c>0,a>0$
% \item $a+b=c,\;c>0,\;a>0$
 
% \item \Large{$$a + b \ge \sqrt{ab}$$}\large
 
 
% \item $\dfrac{234}{3456}$
 
% \item Окружность $\Omega$ касается угла $A$ в точках $B$ и $C$. Докажите, что центр вписанной в треугольник $ABC$ окружности лежит на $\Omega$.
 
% \item $\sqrt[3]{2}$
 
\end{enumerate}
\end{document}